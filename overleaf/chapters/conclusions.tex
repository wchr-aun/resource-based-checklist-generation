% The body of your dissertation, before the references and any appendices,
% \emph{must} finish by page~40. The introduction, after preliminary material,
% should have started on page~1.

% You may not change the dissertation format (e.g., reduce the font size, change
% the margins, or reduce the line spacing from the default 1.5 spacing). Be
% careful if you copy-paste packages into your document preamble from elsewhere.
% Some \LaTeX{} packages, such as \texttt{fullpage} or \texttt{savetrees}, change
% the margins of your document. Do not include them!

% Over-length or incorrectly-formatted dissertations will not be accepted and you
% would have to modify your dissertation and resubmit. You cannot assume we will
% check your submission before the final deadline and if it requires resubmission
% after the deadline to conform to the page and style requirements you will be
% subject to the usual late penalties based on your final submission time.

% Future Work blah blah
% % workflow user side, security, authentication, authorisation, integration with workflowFM, doing more facility features

% what is this project, objectives, scope + design .1 and .2, eval and results, contribution/beneficiaries.
% future work

The objectives of this project were to design and implement a prototype of a resource-based checklist generation tool for WorkflowFM. Checklist generation tools are essential to workflow management frameworks because they can ensure consistency and completeness, and reduce mistakes. A resource-based checklist generation tool is a tool which allows users to create a checklist based on workflow processes. Being resource-based not only ensures consistency and completeness of tasks but also guides users to develop correct workflows, which improves the overall efficiency of business processes. A checklist generated by this tool can retrieve and display any related information and elements through user interfaces on a hosted web-based application. This project was divided into three chapters: design, implementation, and evaluation.

In the design chapter, we designed this prototype based on the requirements from previous work, the product owner, and relevant tools. The design includes functional specification, software structure, and interface navigation. The functional specification is the features of the prototype from interpreting the requirements, which contains six conventional features and three novel assisting features. We then designed the software structure based on the specification. The software structure contains three parts: backend, frontend, and database. Lastly, we designed the interface navigation that includes eight interactable screens.
These designs served as the foundation for the implementation in the following chapter.

In the implementation chapter, we developed the prototype based on the designs. Additionally, we chose the technologies used in this project, which include TypeScript, Next.js, TailwindCSS, Redux, Scala, Akka-HTTP, PostgreSQL, GitHub, GitHub Actions, Vercel, and Heroku. The implementation was separated into two parts: frontend and backend. Before moving on to evaluating the prototype, we also tested the prototype with various healthcare examples.

In the evaluation chapter, we set up a user evaluation to measure the usability and accuracy of the system. The user evaluation contained two scenarios and three tasks. The first scenario served as a warm-up instance with instructions to acquaint participants with the system. The first scenario included two identical tasks, one of which prohibited the usage of assisting features and the other of which permitted them. The second scenario was provided without any instructions to see if participants would be able to use the system without any guidance. The results demonstrated that participants were able to complete the tasks with remarkably high accuracy despite the system being considerably complex. It is also worth mentioning that the time spent between the two tasks in the first scenario was incredibly different (15 min to 5 min). This could imply that the assisting features significantly help users create checklists faster.

With that said, we are able to design and implement a prototype of a resource-based checklist generation tool with high accuracy scores in the evaluation.
This directly benefits the research team and the businesses using WorkflowFM with a newly developed checklist generation prototype. In addition, this project sets a new baseline for a resource-based checklist generation tool for other researchers and encourages them to develop more innovations for future work, such as the checklist execution and the integration with WorkflowFM.

For the source codes of this work, please refer to \href{https://github.com/wchr-aun/resource-based-checklist-generation}{https://github.com/wchr-aun/resource-based-checklist-generation}
