% The preliminary material of your report should contain:
% \begin{itemize}
% \item
% The title page.
% \item
% An abstract page.
% \item
% Declaration of ethics and own work.
% \item
% Optionally an acknowledgements page.
% \item
% The table of contents.
% \end{itemize}

% As in this example \texttt{skeleton.tex}, the above material should be
% included between:
% \begin{verbatim}
% \begin{preliminary}
%     ...
% \end{preliminary}
% \end{verbatim}
% This style file uses roman numeral page numbers for the preliminary material.

% The main content of the dissertation, starting with the first chapter,
% starts with page~1. \emph{\textbf{The main content must not go beyond page~40.}}

% The report then contains a bibliography and any appendices, which may go beyond
% page~40. The appendices are only for any supporting material that's important to
% go on record. However, you cannot assume markers of dissertations will read them.

% You may not change the dissertation format (e.g., reduce the font size, change
% the margins, or reduce the line spacing from the default 1.5 spacing). Be
% careful if you copy-paste packages into your document preamble from elsewhere.
% Some \LaTeX{} packages, such as \texttt{fullpage} or \texttt{savetrees}, change
% the margins of your document. Do not include them!

% Over-length or incorrectly-formatted dissertations will not be accepted and you
% would have to modify your dissertation and resubmit. You cannot assume we will
% check your submission before the final deadline and if it requires resubmission
% after the deadline to conform to the page and style requirements you will be
% subject to the usual late penalties based on your final submission time.

% \section{Using Sections}

% Divide your chapters into sub-parts as appropriate.

% \section{Citations}

% Citations (such as \cite{P1} or \cite{P2}) can be generated using
% \texttt{BibTeX}. For more advanced usage, we recommend using the \texttt{natbib}
% package or the newer \texttt{biblatex} system.

% These examples use a numerical citation style. You may use any consistent
% reference style that you prefer, including ``(Author, Year)'' citations.


% objective, users, benefits, scope of work, relevant works, previous works, future works, project background, techniques


\section{Problem Statement and Significance}

-Probably what a workflow is-

Business process management (BPM) tools are software to help design, automate, and manage systematic workflows for various business areas. The ease with which repetitive task automation can be implemented using BPM tools is one of its primary benefits. That is because it requires no prior coding experience to develop a workflow in BPM tools. However, the tools are sufficiently customisable for a more experienced user to create a more complex workflow.

% WorkflowFM \cite{papapanagiotou2017workflowfm} is a BPM tool developed and maintained by a research team at the University of Edinburgh that allows people in various industries to build their automated workflows from scratch. What distinguishes WorkflowFM from other competitors is that the models are based on the data structure of the input and output of each task. This further reduces the complexity when generating codes since the system can rely on the structure of the resources rather than the arbitrary input and output defined by users. Moreover, WorkflowFM can generate code required for the control flow automatically. Despite the advantages provided by WorkflowFM, it is still in a very early stage with limited features.

WorkflowFM \cite{papapanagiotou2017workflowfm} is a BPM tool developed by a research team at the University of Edinburgh.
% WorkflowFM's main goal is to help create workflows strictly to the data models within the system. That means each process needs to strictly follow the data models provided within the system and no arbitrary information are allowed.
The objective of WorkflowFM is to assist in developing workflows that rigorously adhere to the system's data models. This means that no random information is permitted and that each process must strictly follow to the data models provided within the system.
% Compared to other competitors which do not rigorously conform to the data models, WorkflowFM is easier to ge
Nevertheless, WorkflowFM is still in a very early stage with limited features.

% Automated workflows are complicated processes that often require human interaction in some form. However, humans cannot interact with automated workflows without some input interface. Forms or checklist provide a straightforward interface for human to workflows. Introducing a checklist form to BPM tools enables the integration of human interactions in between automated tasks. Most of the well-known BPM tools all come with some kind of checklist form defining. Bizagi \cite{bizagi}, for example, can define forms within its automated flows and connect the input to the following tasks via a shared database. However, WorkflowFM lacks this feature.

% A checklist is a type of job aid used to reduce failure by compensating for potential limits of human memory and attention. It helps to ensure consistency and completeness in carrying out a task. A basic example is the "to do list".[1] A more advanced checklist would be a schedule, which lays out tasks to be done according to time of day or other factors. A primary task in checklist is documentation of the task and auditing against the documentation.[2]

% Use of a written checklist can reduce any tendency to avoid, omit or neglect important steps in any task

% -Probably what a checklist is and why it is important-
A checklist is a series of items used to reduce mistakes and failure from human errors \cite{whatischecklist}.
In an automated workflow where processes can often be complicated and require human interaction to proceed, checklists could provide a straightforward input forms for humans to interact. Not only checklists enable human interactions to workflows, but they also prevent mistakes from occurring.

-Why having checklist in WorkflowFM is significance-

% -What I am going to do in this project-

% -What scope of the project I am looking to do-

% -What end results of the project should look like-


\section{Objectives}


\section{Scope of Work}